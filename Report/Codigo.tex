\chapter{Codigo}

\section{Problem}

Los problemas a optimizar siguen una mista estructura. Dado esto, se optó por diseñar una clase que generaliza el problema y permite y facilita la definición de nuevos problemas.

El código \ref{lst:problem} incluye los siguientes métodos los cuales deben de ser implementados por sus clases hijas:
\begin{itemize}
	\item \textit{generateInformation}: Función que define la información necesaria para calcular la función objetivo. La información adicional depende del problema.
	\item \textit{objective}: Función que calcula la utilidad de la solución, si es un problema de minimización tiene que devolver con el signo opuesto. 
	\item \textit{generateInitialSolution}: Función que genera una solución.
	\item \textit{getRandomNeighbour}: Obtiene un vecino de forma aleatoria.
	\item \textit{getNextNeighbour}: Si es necesario manejar indices, esta función permite obtener un vecino tras otro.
	\item \textit{getNeighbours}: Devuelve todos los vecinos.
\end{itemize}

\begin{lstlisting}[style=pythonstyle, label={lst:problem} ,caption={Clase \textit{problem}}]
class Problem(ABC):
  def __init__(self,):
  self.information:Any = {}

  @abstractmethod
  def generateInformation(self, *args, **kwargs) -> None: pass

  @abstractmethod
  def objective(self, solution: Any) -> float: pass

  @abstractmethod
  def generateInitialSolution(self) -> Any: pass

  @abstractmethod
  def getRandomNeighbour(self, solution:Any) -> Any: pass

  @abstractmethod
  def getNextNeighbour(self, solution:Any, *args, **kwargs) -> Any: pass

  @abstractmethod
  def getNeighbours(self, solution: Any) -> list: pass

  @abstractmethod
  def printInformation(self) -> None: pass
\end{lstlisting}

\subsubsection{Knapsack problem}

La clase \textit{knapsackProblem} hereda de la clase \textit{Problem} e implementa las siguientes funciones.

La función \ref{lst:kp-gi} genera de manera aleatoria un conjunto de elementos en la mochila con pesos y valores aleatorios en un rango de $[1,10]$. La función \ref{lst:kp-e} calcula la energía del sistema la cual suma todos los pesos y valores de los elementos que se encuentran en la solución, si el peso es menor al capacidad de la mochila entonces devuelve el valor de la mochila, en caso contrario devuelve la diferencia de el peso actual menos la capacidad máxima.

La función \ref{lst:kp-gis} genera soluciones aleatorias de combinaciones y no regresa ninguna de ellas hasta que el peso de la solución sea menor a la capacidad máxima. Finalmente, la función \ref{lst:kp-rn} se encarga de generar un vecino de forma aleatoria, primero selecciona un elemento aleatorio del vector y después lo invierte.  

\begin{lstlisting}[style=pythonstyle, label={lst:kp-gi} ,caption={Función \textit{generateInformation} de Knapsack problem}]
def generate_information(self, items, capacity):
  self.information = {
	"items": items,
	"values": [(random.randint(1,10), random.randint(1, 10)) for _ in range(items)],
	"capacity": capacity
  }
\end{lstlisting}

\begin{lstlisting}[style=pythonstyle, label={lst:kp-e} ,caption={Función \textit{objective} de Knapsack problem}]
def objective(self, solution):
  total_weight = total_value = 0
  for i, selected in enumerate(solution):
    if selected:
      total_weight += self.information["values"][i][0]
  	  total_value += self.information["values"][i][1]
  if total_weight > self.information["capacity"]:
    return self.information["capacity"] - total_weight 
  return total_value
\end{lstlisting}

\begin{lstlisting}[style=pythonstyle, label={lst:kp-gis} ,caption={Función \textit{generateInitialSolution} de Knapsack problem}]
def generateInitialSolution(self):
  return [random.randint(0, 1) for _ in self.information['values']]
\end{lstlisting}

\begin{lstlisting}[style=pythonstyle, label={lst:kp-rn} ,caption={Función \textit{getRandomNeighbour} de Knapsack problem}]
def getRandomNeighbour(self, solution):
  neighbour = solution[:]
  index = random.randint(0, len(solution) - 1)
  neighbour[index] = 1 - int(neighbour[index])
  return neighbour  
\end{lstlisting}

Para el algoritmo genético, no es necesario especificar el resto de funciones.

\subsection{Travel Salesman Problem}

La función \ref{lst:tsp-gi} genera una matriz cuadrada y simétrica de $n$ valores aleatorios en un rango de $[1, 100]$.

La función \ref{lst:tsp-e} calcula la energía del sistema. Dado un vector suma todas las distancias de las ciudades con base en la información generada en \ref{lst:tsp-gi}, el resultado que devuelve es negativo ya que se busca minimizar. La función \ref{lst:tsp-gis} genera un vector de $n$ números consecutivos (representando las ciudades) y cambia las posiciones mediante la función \textit{shuffle}.

Finalmente, la función \ref{lst:tsp-rn} toma dos indices aleatorios diferentes e invierte los valores de dichas posiciones del vector.

\begin{lstlisting}[style=pythonstyle, label={lst:tsp-gi} ,caption={Función \textit{generateInformation} de Travel Salesman Problem}]
def generateInformation(self, cities: int):
  distances = [[0]*cities for _ in range(cities)]

  for i in range(cities):
    for j in range(i, cities):  
      if i == j:
        valor = 0  
      else:
        valor = random.randint(1, 100)
      distances[i][j] = valor
  	  distances[j][i] = valor 

  self.information = {
	"cities" : cities,
	"distances" : distances
  }
\end{lstlisting}

\begin{lstlisting}[style=pythonstyle, label={lst:tsp-e} ,caption={Función \textit{Objective} de Travel Salesman Problem }]
def objective(self, solution):
  distance = 0
  num_cities:int = len(solution)

  for i in range(num_cities):
    current_city = solution[i]
    next_city = solution[(i + 1) % num_cities]  
    distance += self.information['distances'][current_city][next_city]

  return -distance
\end{lstlisting}

\begin{lstlisting}[style=pythonstyle, label={lst:tsp-gis} ,caption={Función \textit{generateInitialSolution} de Travel Salesman Problem}]
def generateInitialSolution(self):
  solution =  list(range(self.information['cities']))
  random.shuffle(solution)
  return solution   
\end{lstlisting}

\begin{lstlisting}[style=pythonstyle, label={lst:tsp-rn} ,caption={Función \textit{getRandomNeighbour} de Travel Salesman Problem}]
def getRandomNeighbour(self, solution):
  neighbour = solution[:]
  i = j = random.randint(0, len(solution) - 1)
  while j == i:
    j = random.randint(0, len(solution) - 1)
  neighbour[i], neighbour[j] = neighbour[j], neighbour[i]

  return neighbour
\end{lstlisting}

\subsection{Sum function Problem}

La función \ref{lst:sfp-gi} unicamente define el tamaño del vector y los rangos de valores. por otro lado, la función \ref{lst:sfp-e} calcula la energía del sistema dada por la suma de los cuadrados, dado que es una función de minimización se invierte el signo.

La función \ref{lst:sfp-gis} genera un vector de $n$ elementos aleatorios en los rangos definidos, mientras que la función \ref{lst:sfp-gn} suma  o resta en uno a un elemento aleatorio del vector (dado que se considera una configuración circular, se ajusta el valor si el nuevo valor no se encuentra en el rango).

\begin{lstlisting}[style=pythonstyle, label={lst:sfp-gi} ,caption={Función \textit{generateInformation} de SumFunctionProblem}]
def generateInformation(self, size: int, min: int, max: int):
  self.information = {
	"size": size,
	"min": min,
	"max": max
  }
\end{lstlisting}

\begin{lstlisting}[style=pythonstyle, label={lst:sfp-e} ,caption={Función \textit{objective} de SumFunctionProblem}]
def objective(self, solution):
  total_sum:float = 0

  for val in solution:
    total_sum += val**2

  return -total_sum
\end{lstlisting}

\begin{lstlisting}[style=pythonstyle, label={lst:sfp-gis} ,caption={Función \textit{generateInitialSolution} de SumFunctionProblem}]
def generateInitialSolution(self):
  solution = [random.randint(self.information['min'], self.information['max']) for _ in range(self.information['size'])]
  return solution
\end{lstlisting}

\begin{lstlisting}[style=pythonstyle, label={lst:sfp-gn} ,caption={Función \textit{getRandomNeighbour} de CEC 2017}]
def getRandomNeighbour(self, solution):
  neighbour = solution[:]
  index = random.randint(0, len(solution) - 1)
  sign = random.choice([-1 , 1])
  neighbour[index] += sign

  if neighbour[index] > self.information['max']:
    neighbour[index] = self.information['min']

  if neighbour[index] < self.information['min']:
	neighbour[index] = self.information['max']

  return neighbour
\end{lstlisting}

\subsection{CEC 2017}

La función \ref{lst:cec-gi} define la función a optimizar, los rangos de valores, el tamaño de la dimensión y la tasa de cambio (ya que la solución es real). Por otro lado, la función \ref{lst:cec-e} calcula la energía del sistema dados los parámetros ya definidos.

La función \ref{lst:cec-gis} genera un vector de la dimensión definida restringida por la información adicional definida, mientras que la función \ref{lst:cec-gn} suma  o resta en \textit{alpha} a un elemento aleatorio del vector.

\begin{lstlisting}[style=pythonstyle, label={lst:cec-gi} ,caption={Función \textit{generateInformation} de CEC 2017}]
def generateInformation(self, function: callable, low:int, high:int, dimention:int, alpha:int):
  self.information = {
	"function": function,
	"low" : low,
	"high": high,
	"dimention": dimention,
	"alpha": alpha
  }
\end{lstlisting}

\begin{lstlisting}[style=pythonstyle, label={lst:cec-e} ,caption={Función \textit{objective} de CEC 2017}]
def objective(self, solution):
  return - self.information["function"]([solution])[0]
\end{lstlisting}

\begin{lstlisting}[style=pythonstyle, label={lst:cec-gis} ,caption={Función \textit{generateInitialSolution} de CEC 2017}]
def generateInitialSolution(self):
  solution = np.random.uniform(low=self.information["low"],
    high=self.information["high"],
    size=self.information["dimention"]).tolist()

  return solution
\end{lstlisting}

\begin{lstlisting}[style=pythonstyle, label={lst:cec-gn} ,caption={Función \textit{getRandomNeighbour} de CEC 2017}]
def getRandomNeighbour(self, solution):
  neighbour = solution[:]
  index = random.randint(0, len(solution) - 1)
  alpha = random.uniform(-self.information["alpha"], self.information["alpha"])

  neighbour[index] += alpha
  return neighbour
\end{lstlisting}

\section{Metaheuristic}

De igual manera, el se diseña una clase padre \ref{lst:meta} que implemente algunas de las funciones compatibles entre algoritmo genético, \textit{hill climbing} y \textit{simulated annealing}.

Esta clase tiene como parametro la clase \textit{Problem} y define los métodos abstractos \textit{resetProblem} y \textit{optimize} los cuales reinician meta parámetros y evalúan el algoritmo, respectivamente.

Adicionalmente se definen 3 funciones:
\begin{itemize}
	\item \textit{evaluate}: recibe como parámetro una solución y devuelve la aptitud definida por el problema
	
	\item \textit{isBetterSolution}: Compara dos soluciones.
	
	\item \textit{isSameSolution}: Verifica si dos soluciones tiene  la misma aptitud.
\end{itemize}

\begin{lstlisting}[style=pythonstyle, label={lst:meta} ,caption={Clase \textit{Metaheuristic}}]
class Metaheuristic(ABC):
  def __init__(self, problem: Problem):
  self.problem = problem

  @abstractmethod
  def resetProblem(self):
    self.solution = self.problem.generateInitialSolution()
    self.is_best = False

  @abstractmethod
  def optimize(self, *args, **kwargs): pass

  def evaluate(self, state: Any) -> float:
    return self.problem.objective(state)

  def isBetterSolution(self, solution_1: Any, solution_2: Any)-> bool:
    return (self.problem.objective(solution_1) > 
    	self.problem.objective(solution_2)) 

  def isSameSolution(self, solution_1: Any, solution_2: Any) -> bool:
  return (self.problem.objective(solution_1) ==
  	  self.problem.objective(solution_2)) 

  def printSolution(self) -> None:
    print(f'Solution: {self.solution}: {self.evaluate(self.solution)}')

  def getSolution(self) -> Any:
    return self.solution

  def setSolution(self, solution: Any):
    self.solution = solution
\end{lstlisting}

\subsection{Genetic Algorithm}

La clase \textit{GeneticAlgorithm} hereda de la clase \textit{Metaheuristic} e incluye las clases \textit{GASelectionFunctions}, \textit{GACrossoverFunctions}, \textit{GAMutationFunctions} y \textit{GAReplaceFunctions} que se verán mas adelante.

El constructor del código \ref{lst:ga-init} recibe como parametros el problema de optimización y el tamaño de la población. Adicionalmente crea los objetos de las funciones para poder ser accedidas desde esa misma clase y llama la función \ref{lst:ga-rp}, la cual predefine las funciones del algoritmo e inicializa la población.

La función \ref{lst:ga-o} se encarga de encontrar la mejor solución realizando las operaciones de selección, cruza, mutación y reemplazo. Nótese que esta función recibe como parámetro \textit{stationary} de tipo \textit{float} el cual define el tamaño de la población que se mantiene estacionaria y cual evoluciona. A la parte que evoluciona se le aplican las operaciones de selección, cruza y mutación para al final aplicar la operación de reemplazo a la población en general.

Adicionalmente, la función \ref{lst:ga-set} es el \textit{setter} de la función a utilizar, el resto de funciones tienen la misma estructura. Se utiliza \textit{partial} ya que existen funciones que requieren de parámetros adicionales.

\begin{lstlisting}[style=pythonstyle, label={lst:ga-init} ,caption={Constructor  \textit{GeneticAlgorithm}}]
	class Metaheuristic(ABC):
def __init__(self, problem, population_size: int = 16):
  super().__init__(problem)
  self.population_size = population_size
  self.selection_functions = Selection
  self.crossover_functions = Crossover
  self.mutation_functions = Mutation
  self.replace_functions = Replace
  self.resetProblem()
\end{lstlisting}

\begin{lstlisting}[style=pythonstyle, label={lst:ga-rp} ,caption={Función  \textit{resetProblem}}]
def resetProblem(self):
super().resetProblem()
self.population = [self.problem.generateInitialSolution() for _ in range(self.population_size)]
self.selection = Selection.tournament
self.crossover = Crossover.onePoint
self.mutation = Mutation.singlePoint
self.replace = Replace.random
\end{lstlisting}

\begin{lstlisting}[style=pythonstyle, label={lst:ga-o} ,caption={Función  \textit{optimize}}]
def optimize(self, epochs: int = 1, stationary: float = 0):
  for _ in range(epochs):
    population_sample = random.sample(self.population,self.population_size - int(self.population_size * stationary))

    population_sample = self.selection(population_sample, self.problem.objective)
    population_sample = self.crossover(population_sample)
    population_sample = self.mutation(population_sample, self.problem)
    self.population = self.replace(self.population, population_sample, self.problem.objective) 
\end{lstlisting}

\begin{lstlisting}[style=pythonstyle, label={lst:ga-set} ,caption={Función  \textit{optimize}}]
def setSelection(self, selection, selection_rate = None):
  if selection_rate:
    self.selection = partial(selection, selection_rate = selection_rate)
  else:
    self.selection = selection
\end{lstlisting}

\subsection{Selection functions}

Las funciones de selección reciben como parametros la población de la cual se seleccionan los padres, la función objetivo y retorna el conjunto de individuos de padres de la siguiente generación.

 A conitnuación se muestran tres operadores de selección: 
 \begin{itemize}
 	\item \textit{tournament}. Recibe un parámetro adicional que define la cantidad de individuos que participaran en cada torneo.
 	\item \textit{proportional}
 	\item \textit{negativeAssortativeMating}. Aunque recibe como parámetro la función objetivo no es requerida, ya que es at función busca emparejar aquellos individuos mas diferentes.
 \end{itemize}
 
\begin{lstlisting}[style=pythonstyle, label={lst:ga-t} ,caption={Función  \textit{tournament}}]
@staticmethod
def tournament(population, objective, selection_rate=0.2):
  population_size = len(population)
  parents = []

  for _ in range(population_size):
    k = max(2, math.ceil(population_size * selection_rate))
    candidates = random.sample(population, min(k, population_size))
    scores = [objective(ind) for ind in candidates]
    best = candidates[scores.index(max(scores))]
    parents.append(best)

  return parents    
\end{lstlisting}

\begin{lstlisting}[style=pythonstyle, label={lst:ga-p} ,caption={Función  \textit{proportional}}]
@staticmethod
def proportional(population, objective):
  population_size = len(population)
  parents = []

  objectives = SelectionFunctions._linealDisplacement(population, objective)

  total = sum(objectives)
  probabilities = [ objective/total for objective in objectives ]

  proportions = [ probabilities[0] ]
  for i in range(1, len(probabilities)):
    proportions.append(probabilities[i] + proportions[i - 1])

  for _ in range(population_size):
    pos = random.random()
    for i in range(len(proportions)):
      if (pos <= proportions[i]):
        parents.append(population[i])
        break

  return parents
\end{lstlisting}

\begin{lstlisting}[style=pythonstyle, label={lst:ga-nam} ,caption={Función  \textit{negativeAssortativeMating}}]
@staticmethod
def negativeAssortativeMating(population, objective):
  population_size = len(population)
  parents = []

  for _ in range(int(population_size/2)):
    individual = random.choice(population)
    distances = [ SelectionFunctions.distance(individual, test) for test in population ]

    parents.append(individual)
    parents.append(population[distances.index(max(distances))])

  return parents
\end{lstlisting}

\subsection{Crossover functions}

A continuación, se muestra la implementación de tres algoritmos de cruza:
\begin{itemize}
	\item \textit{onePoint} para soluciones binarias.
	\item \textit{arithmetic} para soluciones reales.
	\item \textit{uniformOrderBased} para soluciones de permutación.
\end{itemize}

\begin{lstlisting}[style=pythonstyle, label={lst:ga-op} ,caption={Función  \textit{onePoint}}]
def onePoint(population:list):
  generation:list = []    
  population_size = len(population)
  for i in range(0, population_size, 2):
    parent1 = population[i]
    parent2 = population[i + 1]

    point:int = random.randint(1, population_size - 1)

    child1 = parent1[:point] + parent2[point:]
    child2 = parent2[:point] + parent1[point:]
    generation.extend([child1, child2])

  return generation
\end{lstlisting}

\begin{lstlisting}[style=pythonstyle, label={lst:ga-a} ,caption={Función  \textit{arithmetic}}]
@staticmethod
def arithmetic(population: list, crossover_rate: float = None):
  generation = []
  population_size = len(population)

  for i in range(0, population_size - 1, 2):
    parent1 = population[i]
    parent2 = population[i + 1]

    a = crossover_rate if crossover_rate is not None else random.random()

    child1 = [a * x + (1 - a) * y for x, y in zip(parent1, parent2)]
    child2 = [(1 - a) * x + a * y for x, y in zip(parent1, parent2)]

    generation.extend([child1, child2])

  return generation
\end{lstlisting}

\subsection{Mutation functions}

Se utilizo unicamente una función de mutación que busca un vecino inmediato de la solución que se le otorgue dada una probabilidad.

\begin{lstlisting}[style=pythonstyle, label={lst:ga-a} ,caption={Función  \textit{arithmetic}}]
@staticmethod
def singlePoint(population:list, problem, mutation_rate: float = 0.2):
  mutations: list  = []
  for individual in population:
    if random.random() <= mutation_rate:
      mutations.append(problem.getRandomNeighbour(individual))
    else:
      mutations.append(individual)

  return mutations
\end{lstlisting}

\subsection{Replace functions}



